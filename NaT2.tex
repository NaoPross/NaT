\documentclass[a4paper, twocolumn]{article}

%
% Packages
%

%% styling for this document
\usepackage{tex/docstyle}

%% Test
\usepackage{blindtext}

%
% Metadata
%

\title{Nachrichtentechnik 2}
\author{
  Naoki Pross
}
\date{\today}

%
% Macros
%

\newcommand{\overbar}[1]{\mkern 1.5mu\overline{\mkern-1.5mu#1\mkern-1.5mu}\mkern 1.5mu}

%
% Document
%

\begin{document}
\maketitle
\tableofcontents

\section{Probability and Random Variables}
\begin{defn}[Random experiment]
  The experiment is called \emph{random} if its outcome cannot be predicted.
\end{defn}

\begin{defn}[Sample space and events]
  The set of all possible outcomes of a random experiment is called the
  \emph{sample space} \(S\). An \(\lambda \in S\) is called a \emph{sample
  point}. A set of sample points \(A \subset S\) is called an \emph{event}.
\end{defn}

\begin{defn}[Auxiliary definitions]
  The \emph{complement} of an event \(A\) is \(\overbar{A} = \{\lambda \in S :
  \lambda \notin A\}\). The \emph{union} of two events \(A\) and \(B\) is
  \(A\cup B = \{\lambda \in S : \lambda \in A \vee \lambda \in B\}\), similarly
  the \emph{intersection} is \(A\cap B = \{\lambda \in S : \lambda \in A \wedge
  \lambda \in B\}\).
\end{defn}

% TODO: rest of chapter

\section{Random Processes}

Previously we defined a random variable as a function \(X: S \to
E\subset\mathbb{R}\) that assigns a \emph{number} to each event. We will now
extend this concept.

\begin{defn}[Random Process]
  For a random experiment with sample space \(S\), we assign to every event
  \(\lambda \in S\) a \emph{function} of time. The \emph{random process} \(X(t,
  \lambda)\) is effectively \(X : \mathbb{R} \times S \to E \subset\mathbb{R}\).
\end{defn}

\begin{remark}
  Notice that for a fixed \(\lambda_i\), \(X(t,\lambda_i) = X(t)\) is indeed a
  function of time. Conversely for a fixed time \(t_i\), \(X(t_i, \lambda)\) is
  a a random variable. Only when both a \(\lambda_i\) and \(t_i\) are given
  \(X(t_i, \lambda_i)\) is a number.
\end{remark}

\begin{defn}[Distribution]


\end{defn}

\end{document}
